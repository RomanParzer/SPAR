% Options for packages loaded elsewhere
\PassOptionsToPackage{unicode}{hyperref}
\PassOptionsToPackage{hyphens}{url}
\PassOptionsToPackage{dvipsnames,svgnames,x11names}{xcolor}
%
\documentclass[
  article]{jss}

\usepackage{amsmath,amssymb}
\usepackage{iftex}
\ifPDFTeX
  \usepackage[T1]{fontenc}
  \usepackage[utf8]{inputenc}
  \usepackage{textcomp} % provide euro and other symbols
\else % if luatex or xetex
  \usepackage{unicode-math}
  \defaultfontfeatures{Scale=MatchLowercase}
  \defaultfontfeatures[\rmfamily]{Ligatures=TeX,Scale=1}
\fi
\usepackage{lmodern}
\ifPDFTeX\else  
    % xetex/luatex font selection
\fi
% Use upquote if available, for straight quotes in verbatim environments
\IfFileExists{upquote.sty}{\usepackage{upquote}}{}
\IfFileExists{microtype.sty}{% use microtype if available
  \usepackage[]{microtype}
  \UseMicrotypeSet[protrusion]{basicmath} % disable protrusion for tt fonts
}{}
\makeatletter
\@ifundefined{KOMAClassName}{% if non-KOMA class
  \IfFileExists{parskip.sty}{%
    \usepackage{parskip}
  }{% else
    \setlength{\parindent}{0pt}
    \setlength{\parskip}{6pt plus 2pt minus 1pt}}
}{% if KOMA class
  \KOMAoptions{parskip=half}}
\makeatother
\usepackage{xcolor}
\setlength{\emergencystretch}{3em} % prevent overfull lines
\setcounter{secnumdepth}{-\maxdimen} % remove section numbering
% Make \paragraph and \subparagraph free-standing
\makeatletter
\ifx\paragraph\undefined\else
  \let\oldparagraph\paragraph
  \renewcommand{\paragraph}{
    \@ifstar
      \xxxParagraphStar
      \xxxParagraphNoStar
  }
  \newcommand{\xxxParagraphStar}[1]{\oldparagraph*{#1}\mbox{}}
  \newcommand{\xxxParagraphNoStar}[1]{\oldparagraph{#1}\mbox{}}
\fi
\ifx\subparagraph\undefined\else
  \let\oldsubparagraph\subparagraph
  \renewcommand{\subparagraph}{
    \@ifstar
      \xxxSubParagraphStar
      \xxxSubParagraphNoStar
  }
  \newcommand{\xxxSubParagraphStar}[1]{\oldsubparagraph*{#1}\mbox{}}
  \newcommand{\xxxSubParagraphNoStar}[1]{\oldsubparagraph{#1}\mbox{}}
\fi
\makeatother


\providecommand{\tightlist}{%
  \setlength{\itemsep}{0pt}\setlength{\parskip}{0pt}}\usepackage{longtable,booktabs,array}
\usepackage{calc} % for calculating minipage widths
% Correct order of tables after \paragraph or \subparagraph
\usepackage{etoolbox}
\makeatletter
\patchcmd\longtable{\par}{\if@noskipsec\mbox{}\fi\par}{}{}
\makeatother
% Allow footnotes in longtable head/foot
\IfFileExists{footnotehyper.sty}{\usepackage{footnotehyper}}{\usepackage{footnote}}
\makesavenoteenv{longtable}
\usepackage{graphicx}
\makeatletter
\def\maxwidth{\ifdim\Gin@nat@width>\linewidth\linewidth\else\Gin@nat@width\fi}
\def\maxheight{\ifdim\Gin@nat@height>\textheight\textheight\else\Gin@nat@height\fi}
\makeatother
% Scale images if necessary, so that they will not overflow the page
% margins by default, and it is still possible to overwrite the defaults
% using explicit options in \includegraphics[width, height, ...]{}
\setkeys{Gin}{width=\maxwidth,height=\maxheight,keepaspectratio}
% Set default figure placement to htbp
\makeatletter
\def\fps@figure{htbp}
\makeatother

\usepackage{orcidlink,thumbpdf,lmodern}

\newcommand{\class}[1]{`\code{#1}'}
\newcommand{\fct}[1]{\code{#1()}}
\makeatletter
\@ifpackageloaded{caption}{}{\usepackage{caption}}
\AtBeginDocument{%
\ifdefined\contentsname
  \renewcommand*\contentsname{Table of contents}
\else
  \newcommand\contentsname{Table of contents}
\fi
\ifdefined\listfigurename
  \renewcommand*\listfigurename{List of Figures}
\else
  \newcommand\listfigurename{List of Figures}
\fi
\ifdefined\listtablename
  \renewcommand*\listtablename{List of Tables}
\else
  \newcommand\listtablename{List of Tables}
\fi
\ifdefined\figurename
  \renewcommand*\figurename{Figure}
\else
  \newcommand\figurename{Figure}
\fi
\ifdefined\tablename
  \renewcommand*\tablename{Table}
\else
  \newcommand\tablename{Table}
\fi
}
\@ifpackageloaded{float}{}{\usepackage{float}}
\floatstyle{ruled}
\@ifundefined{c@chapter}{\newfloat{codelisting}{h}{lop}}{\newfloat{codelisting}{h}{lop}[chapter]}
\floatname{codelisting}{Listing}
\newcommand*\listoflistings{\listof{codelisting}{List of Listings}}
\makeatother
\makeatletter
\makeatother
\makeatletter
\@ifpackageloaded{caption}{}{\usepackage{caption}}
\@ifpackageloaded{subcaption}{}{\usepackage{subcaption}}
\makeatother
\makeatletter
\@ifpackageloaded{tcolorbox}{}{\usepackage[skins,breakable]{tcolorbox}}
\makeatother
\makeatletter
\@ifundefined{shadecolor}{\definecolor{shadecolor}{rgb}{.97, .97, .97}}{}
\makeatother
\makeatletter
\makeatother
\makeatletter
\ifdefined\Shaded\renewenvironment{Shaded}{\begin{tcolorbox}[boxrule=0pt, breakable, sharp corners, borderline west={3pt}{0pt}{shadecolor}, interior hidden, enhanced, frame hidden]}{\end{tcolorbox}}\fi
\makeatother

\ifLuaTeX
  \usepackage{selnolig}  % disable illegal ligatures
\fi
\usepackage{bookmark}

\IfFileExists{xurl.sty}{\usepackage{xurl}}{} % add URL line breaks if available
\urlstyle{same} % disable monospaced font for URLs
\hypersetup{
  pdftitle={: Sparse Projected Averaged Regression in },
  pdfauthor={Roman Parzer; Peter Filzmoser; Laura Vana Gür},
  pdfkeywords={random projection, variable screening, ensemble
learning, R},
  colorlinks=true,
  linkcolor={blue},
  filecolor={Maroon},
  citecolor={Blue},
  urlcolor={Blue},
  pdfcreator={LaTeX via pandoc}}


%% -- Article metainformation (author, title, ...) -----------------------------

%% Author information
\author{Roman Parzer\\TU Wien \And Peter Filzmoser\\TU Wien \AND Laura
Vana Gür\\TU Wien}
\Plainauthor{Roman Parzer, Peter Filzmoser, Laura Vana
Gür} %% comma-separated

\title{\pkg{SPAR}: Sparse Projected Averaged Regression in \proglang{R}}
\Plaintitle{: Sparse Projected Averaged Regression
in} %% without formatting

%% an abstract and keywords
\Abstract{\pkg{SPAR} is a package for building ensembles of predictive
generalized linear models (GLMs) with high-dimensional (HD) predictors
in \proglang{R} by making use of probabilistic variable screening and
random projection tools. The package design is focused on extensibility,
where the screening and random projections are implemented as classes
with convenient constructor functions, allowing users to easily
implement new procedures.}

%% at least one keyword must be supplied
\Keywords{random projection, variable screening, ensemble
learning, \proglang{R}}
\Plainkeywords{random projection, variable screening, ensemble
learning, R}

%% publication information
%% NOTE: Typically, this can be left commented and will be filled out by the technical editor
%% \Volume{50}
%% \Issue{9}
%% \Month{June}
%% \Year{2012}
%% \Submitdate{2012-06-04}
%% \Acceptdate{2012-06-04}
%% \setcounter{page}{1}
%% \Pages{1--xx}

%% The address of (at least) one author should be given
%% in the following format:
\Address{
Roman Parzer\\
Computational Statistics (CSTAT) ~Institute of Statistics and
Mathematical Methods in Economics\\
Karlsplatz 4\\
Vienna Austria\\
E-mail: \email{Roman.Parzer@tuwien.ac.at}\\
\\~
Peter Filzmoser\\
\\~
Laura Vana Gür\\
\\~

}

\begin{document}
\maketitle


\section{Introduction}\label{sec-intro}

\pkg{SPAR} is a package for building predictive generalized linear
models (GLMs) with high-dimensional (HD) predictors in \proglang{R}. In
package \pkg{SPAR}, probabilistic variable screening and random
projection of the predictors are performed to obtain an ensemble of
GLMs, which are then averaged to obtain predictions in an
high-dimensional regression setting.

Random projection is a computationally-efficient method which linearly
maps a set of points in high dimensions into a much lower-dimensional
space while approximately preserving pairwise distances. For very large
\(p\), random projection can suffer from noise accumulation, as too many
irrelevant predictors are being considered for prediction purposes
\citep{Dunson2020TargRandProj}. Therefore, screening out irrelevant
variables before performing the random projection is advisable in order
to tackle this issue. The screening can be performed in a probabilistic
fashion, by randomly sampling covariates for inclusion in the model
based on probabilities proportional to an importance measure (as opposed
to random subspace sampling employed in e.g., random forests). Finally,
in practice, the information from multiple such screening and
projections can be combined by averaging, in order to reduce the
variance introduced by the random sampling (of both projections and
screening indicators) \citep{Thanei2017RPforHDR}.

Several packages which provide functionality for random projections are
available for \proglang{R}. Package \pkg{RandPro}
\citep{RandProR, SIDDHARTH2020100629} allows for four different random
projection matrices to be applied to the predictor matrix before
employing one of \(k\)\textasciitilde nearest neighbor, support vector
machine or naive Bayes classifier. Package \pkg{SPCAvRP}
\citep{SPCAvRPR} implements sparse principal component analysis, based
on the aggregation of eigenvector information from
``carefully-selected'' axis-aligned random projections of the sample
covariance matrix. Package \pkg{RPEnsembleR} \citep{RPEnsembleR}
implements the same idea of ``carefully-selected'' random projections
when building an ensemble of classifiers. For \proglang{Python}
\citet{Python} the \pkg{sklearn.random\_projection} module implements
two types of unstructured random matrix, namely Gaussian random matrix
and sparse random matrix.

On the other hand, there are a multitude of packages dealing with
variable screening on the Comprehensive \proglang{R} Archive Network
(CRAN). The (iterative) sure independence screening procedure and
extensions in \citet{Fan2007SISforUHD}, \citet{Fan2010sisglms},
\citet{fan2010high} are implemented in package \pkg{SIS} \citep{SISR},
which also provides functionality for estimating a penalized generalized
linear model or a cox regression model for the variables picked by the
screening procedure.

Package \pkg{VariableScreening} \citep{pkg:VariableScreening} implements
screening for iid data, varying-coefficient models, and longitudinal
data using different screening methods: Sure Independent Ranking and
Screening -- which ranks the predictors by their correlation with the
rank-ordered response (SIRS), Distance Correlation Sure Independence
Screening -- a non-parametric extension of the correlation coefficient
(DC-SIS), MV Sure Independence Screening -- using the mean conditional
variance measure (MV-SIS).

A collection of model-free screening techniques such as SIRS, DC-SIS,
MV-SIS, the fused Kolmogorov filter \citep{mai2015fusedkolmogorov}, the
projection correlation method using knock-off features
\citep{liu2020knockoff}, are provided in package \pkg{MFSIS}
\citep{pkg:MFSIS}. Package \pkg{tilting} \citep{pkg:tilting} implements
an algorithm for variable selection in high-dimensional linear
regression using tilted correlation, which takes into account high
correlations among the variables in a data-driven way. Feature screening
based on conditional distance correlation \citep{wang2015conditional}
can be performed with the \pkg{cdcsis} package \citep{pkg:cdcsis} while
package \pkg{QCSIS} \citep{pkg:QCSIS} implements screening based on
(composite) quantile correlation.

Package \pkg{LqG} \citep{pkg:LqG} provides a group screening procedure
that is based on maximum Lq-likelihood estimation, to simultaneously
account for the group structure and data contamination in variable
screening.

Feature screening using an \(L1\) fusion penalty can be performed with
package \pkg{fusionclust} \citep{pkg:fusionclust}. Package \pkg{SMLE}
\citep{pkg:SMLE} implements joint feature screening via sparse MLE
\citep{SMLE2014} in high-dimensional linear, logistic, and Poisson
models. Package \pkg{TSGSIS} \citep{pkg:TSGSIS} provides a
high-dimensional grouped variable selection approach for detecting
interactions that may not have marginal effects in high dimensional
linear and logistic regression \citep{10.1093/bioinformatics/btx409}.

Package \pkg{RaSEn} \citep{pkg:RaSEn} implements the RaSE algorithm for
ensemble classification and classification problems, where random
subspaces are generated and the optimal one is chosen to train a weak
learner on the basis of some criterion. Various choices of base
classifiers are implemented, for instance, linear discriminant analysis,
quadratic discriminant analysis, k-nearest neighbor, logistic or linear
regression, decision trees, random forest, support vector machines. The
selected percentages of variables can be employed for variable
screening.

Package \pkg{Ball} \citep{pkg:ball} provides functionality for variable
screening using ball statistics, which is appropriate for shape,
directional, compositional and symmetric positive definite matrix data.

Package \pkg{BayesS5} \citep{pkg:BayesS5} implements Bayesian variable
selection using simplified shotgun stochastic search algorithm with
screening \citep{shin2017scalablebayesianvariableselection} while
package \pkg{bravo} \citep{pkg:bravo} implements the Bayesian iterative
screening method proposed in
\citep{wang2021bayesianiterativescreeningultrahigh}.

The rest of the paper is organized as follows: Section~\ref{sec-models}
provides the methodological details of the implemented algorithm. The
package is described in Section~\ref{sec-software}.
Section~\ref{sec-illustrations} contains two examples of employing the
package on real data sets. Finally, Section~\ref{sec-conclusion}
concludes.

\section{Methods}\label{sec-models}

\subsection{Variable screening}\label{variable-screening}

The general idea of variable screening is to select a (small) subset of
variables, based on some marginal utility measure for the predictors,
and disregard the rest for further analysis. In their seminal work on
sure independence screening (SIS), \citet{Fan2007SISforUHD} propose to
use the vector of marginal empirical correlations
\(\hat\alpha=(\alpha_1,\ldots ,\alpha_p)'\in\mathbb{R}^p,\alpha_j=\text{Cor}(X_{.j},y)\)
for variable screening in a linear regression setting by selecting the
variable set \(\mathcal{A}_\gamma = \{j\in [p]:|w_j|>\gamma\}\)
depending on a threshold \(\gamma>0\), where \([p]=\{1,\dots,p\}\).
Under certain technical conditions, where \(p\) grows exponentially with
\(n\), they show that this procedure has the \emph{sure screening
property} \[
\mathbb{P}(\mathcal{A} \subset \mathcal{A}_{\gamma_n})\to 1 \text{ for } n\to \infty
\] with an explicit exponential rate of convergence, where
\(\mathcal{A}=\{j\in[p]:\beta_j\neq 0\}\) is the set of truly active
variables. These conditions imply that \(\mathcal{A}\) and
\(\mathcal{A}_{\gamma_n}\) contain less than \(n\) variables. One of the
critical conditions is that on the population level for some fixed
\(i\in[n]\),
\(\min_{j\in\mathcal{A}}|\text{Cov}(y_i/\beta_j,x_{ij})| \geq c\) for
some constant \(c>0\), which rules out practically possible scenarios
where an important variable is marginally uncorrelated to the response.
\citet{Fan2010sisglms} extend the approach to GLMs, where the screening
is performed based on the log-likelihood of the GLM containing only
\(X_j\) as a predictor:
\(\hat\alpha_j=: \text{min}_{{\beta_j}\in\mathbb{R}}\sum_{i=1}^n -\ell(\beta;y_i,x_{ij})\).

A rich stream of literature focuses on developing semi- or
non-parametric alternatives to SIS which should be more robust to model
misspecification. For linear regression, approaches include using the
ranked correlation \citep{zhu2011model}, (conditional) distance
correlation \citep{li2012feature, wang2015conditional}. or quantile
correlation \citep{ma2016robust}. For GLMs, \citet{fan2011nonparametric}
extend \citet{Fan2010sisglms} by fitting a generalized additive model
with B-splines. Further extensions for discrete (or categorical)
outcomes include the fused Kolmogorov filter \citep{mai2013kolmogorov},
using the mean conditional variance, i.e., the expectation in \(X_j\) of
the variance in the response of the conditional cumulative distribution
function \(\Prob(X\leq x|Y)\) \citep{cui2015model}.
\citet{ke2023sufficient} propose a model free method where the
contribution of each individual predictor is quantified marginally and
conditionally in the presence of the control variables as well as the
other candidates by reproducing-kernel-based \(R^2\) and partial \(R^2\)
statistics.

To account for multicollinearity among the predictors, which can cause
relevant predictors to be marginally uncorrelated with the response,
various extensions have been proposed. In a linear regression setting,
\citet{Wang2015HOLP} propose employing the ridge estimator when the
penalty term converges to zero while \citet{cho2012high} propose using
the tilted correlation, i.e., the correlation of a tilted version of
\(X_j\) with \(y\). For discrete outcomes, joint feature screening
\citet{SMLE2014} has been proposed.

\subsection{Random projection}\label{random-projection}

The random projection method relies on the Johnson-Lindenstrauss (JL)
lemma \citep{JohnsonLindenstrauss1984}, which asserts that there exists
a random map \(\Phi\in \mathbb{R}^{m \times p}\) that embeds any set of
points in \(p\)-dimensional Euclidean space collected in the rows of
\(X\in \mathbb{R}^{n\times p}\) into a \(m\)-dimensional Euclidean space
with \(m< \mathcal{O}(\log n/\varepsilon^2)\) so that all pairwise
distances are maintained within a factor of \(1 \pm \varepsilon\), for
any \(0 <\varepsilon< 1\).

The random map \(\Phi\) should be chosen such that it fulfills certain
conditions \citep[see][]{JohnsonLindenstrauss1984}. The literature
focuses on constructing such matrices either by sampling them from some
``appropriate'' distribution, by inducing sparsity in the matrix and/or
by employing specific fast constructs which lead to efficient
matrix-vector multiplications.

It turns out that the conditions are generally satisfied by nearly all
sub-Gaussian distributions \citep{matouvsek2008variants}. Common choices
are:

\begin{itemize}
\item
  Normal distribution.: \(\Phi_{ij} \overset{iid}{\sim} N(0,1)\)
  \citep{FRANKL1988JLSphere} or
  \(\Phi_{ij} = \begin{cases} {\sim} N(0,1/\sqrt{\psi}) & \text{with prob. } \psi \\ 0 & \text{with prob. } 1 - \psi \end{cases}\)
  \citep{matouvsek2008variants},
\item
  Rademacher distribution
  \citep{ACHLIOPTAS2003JL, LiHastie2006VerySparseRP} \[
  \Phi_{ij} = \begin{cases}
      \pm 1/\sqrt{\psi} & \text{with prob. } \psi/2 \\
      0 & \text{with prob. } 1 - \psi, \quad 0<\psi\leq 1‚
    \end{cases},
  \] where \citet{ACHLIOPTAS2003JL} shows results for \(\psi=1\) and
  \(\psi=1/3\) while \citet{LiHastie2006VerySparseRP} recommend using
  \(\psi=1/\sqrt{p}\) to obtain very sparse matrices.
\end{itemize}

Distributions which are not sub-Gaussian, such as standard Cauchy, have
also been proposed in the literature to tackle scenarios where the data
is high-dimensional, non-sparse, and heavy-tailed by preserving
approximate \(\ell_1\) distances \citep[see e.g.,][]{li2007nonlinear}.

An orthonormalization is usually applied \((\Phi\Phi^\top)^{-1/2}\Phi\).
Orthonormalization can constitute a computational bottleneck for the
random projection method, however, in high-dimensions it can be omitted.

To speed computations, \citet{ailon2009fast} propose the fast Johnson-
Lindenstrauss transform (FJLT), where the random projection matrix is
given by \(\Phi=PHD\) with \(P\) random and sparse,
\(P_{ij} \sim N (0, 1/q)\) with probability \(1/q\) and \(0\) otherwise,
\(H\) the normalized Hadamard (orthogonal) matrix
\(H_{ij} = p^{-1/2}(-1)^{\langle i-1,j-1\rangle}\), where
\(\langle i, j\rangle\) is the dot-product of the \(m\)-bit vectors
\(i\), \(j\) expressed in binary, and \(D = \text{diag}(\pm 1)\) is a
diagonal matrix with random elements \(D_{ii}\).

\citet{Clarkson2013LowRankApproxShort} propose a sparse embedding matrix
\({\Phi=BD}\), where \(B\in\{0,1\}^{m \times p}\) is random binary
matrix and \(D\) is a \(p\times p\) diagonal matrix with
\((D_{ii}+1)/2\sim \text{Unif}\{0,1\}\), and prove that the dimension
\(m\) is bounded by a polynomial in \(r\varepsilon^{-1}\) for
\(0 <\varepsilon< 1\) and \(r\) being the rank of \(X\). While this is
generally larger than that of FJLT, the sparse embedding matrix requires
less time to compute \(\Phi X\) compared to other subspace embeddings.

\citet{parzer2023sparse} propose employing \({D_{ii}=\hat \alpha}\) in
the sparse embedding matrix of \citet{Clarkson2013LowRankApproxShort},
\(\hat \alpha\) is a screening coefficient in the regression such as the
ridge or the HOLP coefficients, and show that the proposed projection
increases the predictive performance in a linear regression setting.

\subsection{Algorithm}\label{sec-algo}

\begin{itemize}
\item
  choose family with corresponding log-likelihood \(\ell(.)\) and link
\item
  standardize predictors \(X:n\times p\)
\item
  calculate screening coefficients \(\hat\alpha\) e.g.,

  \begin{itemize}
  \tightlist
  \item
    ridge:
    \(\hat\alpha=: \text{argmin}_{{\beta}\in\mathbb{R}^p}\sum_{i=1}^n -\ell(\beta;y_i,x_i) + \frac{\varepsilon}{2}\sum_{j=1}^p{\beta}_j^2, \, \varepsilon > 0\)
  \item
    marginal likelihood:
    \(\hat\alpha_j=: \text{min}_{{\beta_j}\in\mathbb{R}}\sum_{i=1}^n -\ell(\beta;y_i,x_{ij})\)
  \end{itemize}
\item
  For \(k=1,\dots,M\) models:

  \begin{itemize}
  \item
    draw \(2n\) predictors with probabilities
    \(p_j\propto |\hat\alpha_j|\) yielding screening index set
    \(I_k=\{j_1^k,\dots,j_{2n}^k\}\subset[p]\)
  \item
    project remaining variables to dim.
    \(m_k\sim \text{Unif}\{\log(p),\dots,n/2\}\) using
    \textbf{projection matrix} \(\Phi_k\) to obtain
    \(Z_k=X_{\cdot I_k}\Phi_k^\top \in \mathbb{R}^{n\times m_k}\):
  \item
    fit a \textbf{GLM} of \(y\) against \(Z_k\) (with small
    \(L_2\)-penalty \cite{glmnet2023}) to obtain estimated coefficients
    \(\gamma^k\in\mathbb{R}^{m_k}\) and
    \(\hat \beta_{I_k}^k=\Phi_k'\gamma^k\),
    \(\hat \beta_{\bar I_k}^k=0\).
  \end{itemize}
\item
  for a given threshold \(\lambda>0\), set all \(\hat\beta_j^k\) with
  \(|\hat\beta_j^k|<\lambda\) to \(0\) for all \(j,k\)
\item
  \textit{Optional:} choose \(M\) and \(\lambda\) via cross-validation
  by repeating steps 1 to 4 for each fold and evaluating a prediction
  performance measure on the withheld fold; and choose \begin{align}
       (M_{\text{best}},\lambda_{\text{best}}) = \text{argmin}_{M,\lambda}\text{Dev}(M,\lambda)
     \end{align}
\item
  combine via \textbf{simple average}
  \(\hat \beta = \sum_{k=1}^M\hat \beta^k / M\)
\item
  \item

  output the estimated coefficients and predictions for the chosen \(M\)
  and \(\lambda\)
\end{itemize}

\section{Software}\label{sec-software}

The package be installed from \proglang{GitHub}

\begin{verbatim}
devtools::install_github("RomanParzer/SPAR")
\end{verbatim}

and loaded by:

\begin{verbatim}
library("SPAR")
\end{verbatim}

In this section we rely for illustration purposes on an example data set
from the package:

\begin{verbatim}
data("example_data", package = "SPAR")
str(example_data)
#> List of 7
#>  $ x     : num [1:200, 1:2000] 1.8302 -0.4251 -1.3893 -0.0947 0.4304 ...
#>  $ y     : num [1:200] -5.64 -23.63 -17.09 13.18 20.91 ...
#>  $ xtest : num [1:100, 1:2000] -0.166 -0.3729 0.0379 0.6774 0.2174 ...
#>  $ ytest : num [1:100] 10.61 -34.1 29.3 35.53 8.67 ...
#>  $ mu    : num 1
#>  $ beta  : num [1:2000] 1 -2 3 2 1 -3 2 3 1 -2 ...
#>  $ sigma2: num 83
\end{verbatim}

\subsection{Main functions and their
arguments}\label{main-functions-and-their-arguments}

The two main functions for fitting the SPAR algorithm are:

\begin{verbatim}
spar(x, y, family = gaussian("identity"), rp = NULL, scrcoef = NULL,
  xval = NULL, yval = NULL, nnu = 20, nus = NULL, nummods = c(20),
  measure = c("deviance", "mse", "mae", "class", "1-auc"),
  inds = NULL, RPMs = NULL, ...)
\end{verbatim}

which implements the algorithm in Section~\ref{sec-algo} without
cross-validation and returns an object of class ``\texttt{spar}'', and

\begin{verbatim}
spar.cv(x, y, family = gaussian("identity"), rp = NULL, scrcoef = NULL,
  nfolds = 10, nnu = 20, nus = NULL, nummods = c(20),
  measure = c("deviance", "mse", "mae", "class", "1-auc"), ...)
\end{verbatim}

which implements the cross-validated procedure and returns an object of
class ``\texttt{spar.cv}''.

The common arguments of these functions are:

\begin{itemize}
\item
  \texttt{x} is an \(n \times p\) numeric matrix of predictor variables.
\item
  \texttt{y} numeric response vector of length \(n\).
\item
  \texttt{family} object from \fct{stats::family}.
\item
  \texttt{rp} an object of type \class{randomprojection}
\item
  \texttt{scrcoef} an object of type \class{screeningcoef}
\item
  \texttt{nnu} is the number of threshold values \(\nu\) which should be
  considered for thresholding; defaults to 20
\item
  \texttt{nus} is an optional vector of \(\nu\) values to be considered
  for thresholding. If it is not provided, is defaults to a grid of
  \texttt{nnu} values. This grid is generated by including zero and
  \texttt{nnu}\(-1\) equally spaced quantiles of the absolute values of
  the estimated coefficients from the marginal models.
\item
  \texttt{nummods} is the number of models to be considered in the
  ensemble; defaults to 20. If a vector is provided, all combinations of
  \texttt{nus} and \texttt{nummods} are considered when choosing the
  optimal \(\nu_\text{best}\) and \(M_\text{best}\).
\item
  \texttt{measure} gives the measure based on which the thresholding
  value \(\nu_\text{opt}\) and the number of models \texttt{M} should be
  chosen on the validation set (for \texttt{spar()}) or in each of the
  folds (in \texttt{spar.cv()}). Defaults to \texttt{"deviance"}, which
  is available for all families. Other options are \texttt{"mse"} or
  \texttt{"mae"} (between responses and predicted means, for all
  families), \texttt{"class"} (misclassification error) and
  \texttt{"1-auc"} (one minus area under the ROC curve) both just for
  binomial family.
\end{itemize}

Furthermore, \texttt{spar()} has the specific arguments:

\begin{itemize}
\item
  \texttt{xval} and \texttt{yval} which are used as validation sets for
  choosing \(\nu_\text{best}\) and \(M_\text{best}\). If not provided,
  \texttt{x} and \texttt{y} will be employed.
\item
  \texttt{inds} is an optional list of length \texttt{max(nummods)}
  containing column index-vectors corresponding to variables that should
  be kept after screening for each marginal model; dimensions need to
  fit those of the dimensions of the provided matrices in \texttt{RPM}.
\item
  \texttt{RPMs} is an optional list of length \texttt{max(nummods)}
  which contains projection matrices to be used in each marginal model.
\end{itemize}

Function \texttt{spar.cv()} has the specific argument \texttt{nfolds}
which is the number of folds to be used for cross-validation. It relies
on \texttt{spar()} as a workhorse, which is called for each fold. The
random projections for each model are held fixed throughout the
cross-validation to reduce the computational burden. This is possible by
calling \texttt{spar()} in each fold with a predefined \texttt{inds} and
\texttt{RPMs} argument.

\subsection{Screening coefficients}\label{screening-coefficients}

The objects for creating screening coefficients are implemented as
\proglang{S}3 classes ``\texttt{screeningcoef}''. These objects are
created by several \texttt{screen\_*} functions, which take \texttt{...}
and a list of controls \texttt{control} as arguments. These functions
return an object of class ``\texttt{screeningcoef}'' which in turn is a
list with three elements:

\begin{itemize}
\item
  \texttt{name},
\item
  \texttt{generate\_scrcoef} -- an \proglang{R} function for generating
  the screening coefficient. This function should have the following
  arguments:

  \begin{itemize}
  \tightlist
  \item
    \texttt{scrcoef}, which is a ``\texttt{screeningcoef}'' object which
    has as attributes all the information passed through \texttt{...}
    ,\\
  \item
    \texttt{data}, which should be a list of \texttt{x} -- the matrix of
    standardized predictors -- and \texttt{y} -- the vector of
    (standardized in the Gaussian case) responses. It returns a vector
    of screening coefficients of length \(p\).
  \end{itemize}
\item
  \texttt{control}, which is the control list in \texttt{screen\_*}.
  These controls are arguments which are needed in
  \texttt{generate\_scrcoef} in order to generate the desired screening
  coefficients.
\end{itemize}

The following screening coefficients are implemented in \pkg{SPAR}:

\begin{itemize}
\item
  \texttt{screen\_marglik()} - computes the screening coefficients by
  the coefficient of \(x_j\) in a univariate GLM using the
  \texttt{stats::glm()} function. It allows to pass a list of controls
  through the \texttt{control} argument to \texttt{stats::glm} such as
  weights, family, offsets.
\item
  \texttt{screen\_corr()} - computes the screening coefficients by the
  correlation between \(y\) and \(x_j\) using the function
  \texttt{stats::cor()}. It allows to pass a list of controls through
  the \texttt{control} argument to \texttt{stats::cor}.
\item
  \texttt{screen\_ridge()} - computes by default the ridge coefficient
  where the penalty \(\lambda\) is very small \citep[see][ for
  motivation]{parzer2024glms}. The function relies on
  \texttt{glmnet::glmnet()} and, while it assumes by default
  \(\alpha = 0\) and a small penalty, it allows to pass a list of
  controls through the \texttt{control} argument to
  \texttt{glmnet::glmnet()} such as \texttt{alpha\ =\ 1}.
\end{itemize}

Further arguments related to screening can be passed through
\texttt{...}, which will then be saved as attributes of the
``\texttt{screeningcoef}'' object. More specifically, the following are
employed in function \texttt{spar()}:

\begin{itemize}
\item
  \texttt{nscreen} integer giving the number of variables to be retained
  after screening; defaults to \(2n\)
\item
  \texttt{split\_data}, boolean which indicates whether 1/4 of the data
  should be used for computing the screening coefficient and the rest
  3/4 for estimating the SPAR algorithm; defaults to \texttt{FALSE}
\item
  \texttt{type} character -- either \texttt{"prob"} (indicating that
  probabilistic screening should be employed) or \texttt{"fixed"}
  (indicating that a fixed set of \texttt{nscreen} variables should be
  employed across the ensemble; defaults to \texttt{type\ =\ "prob"}.
\end{itemize}

For illustration purposes, consider the implemented function
\texttt{screen\_marglik()}, which is used to define a screening
procedure based on the coefficients of univariate marginal GLMs between
each predictor and the response.

\begin{verbatim}
unclass(screen_marglik())
#> $name
#> [1] "screen_marglik"
#> 
#> $generate_scrcoef
#> function(scrcoef, data) {
#>   y <- data$y
#>   x <- data$x
#>   if (is.null(scrcoef$control$family)) {
#>     scrcoef$control$family <- attr(scrcoef, "family")
#>   }
#>   coefs <- apply(x, 2, function(xj){
#>     glm_res <- do.call(function(...) glm(y ~ xj,  ...),
#>                        scrcoef$control)
#>     glm_res$coefficients[2]
#>   })
#>   coefs
#> }
#> <environment: namespace:SPAR>
#> 
#> $control
#> list()
#> 
#> attr(,"split_data")
#> [1] FALSE
#> attr(,"type")
#> [1] "prob"
\end{verbatim}

Function \texttt{generate\_scrcoef\_marglik} defines the generation of
the screening coefficient. It considers the controls in
\texttt{scrcoef\$control} when calling the \texttt{stats::glm()}
function. Given that the proposed framework estimates GLMs for the
marginal models, the \texttt{spar()} function assigns by default its
\texttt{family} argument as an attribute for the \texttt{screeningcoef}
object. In \texttt{generate\_scrcoef\_marglik}, we employ
\texttt{family} attribute of the ``\texttt{screeningcoef}'' object if
the control list argument does not contain any particular family.

For convenience, a constructor function
\texttt{constructor\_screeningcoef()} is provided, which can be used to
create new \texttt{screen\_*} functions.

\subsection{Random projections}\label{random-projections}

Similar to the screening procedure, the objects for creating random
projections are implemented as \proglang{S}3 classes
``\texttt{randomprojection}'' and are created by functions which take
\texttt{...} and a list of controls \texttt{control} as arguments.

These functions return an object of class ``\texttt{randomprojection}''
which in turn is a list with three elements:

\begin{itemize}
\item
  \texttt{name},
\item
  \texttt{generate\_rp\_fun} function for generating the random
  projection matrix. This function should have with arguments \code{rp},
  which is a ``\texttt{randomprojection}'' object, \code{m}, the target
  dimension and a vector of indexes \code{included_vector} which shows
  the column index of the original variables in the \code{x} matrix to
  be projected using the random projection. This is needed due to the
  fact that screening can be employed pre-projection. It returns a
  sparse random projection matrix with \code{m} rows and
  \code{length(included_vector)} columns.
\item
  \texttt{update\_data\_rp} function with attributes relying with
  information from the data. This is relevant for the data driven random
  projections. This function should have with arguments \code{rp}, which
  is a randomprojection object and \code{data}, which is a list
  containing \texttt{x} (the matrix of predictors used as input in
  \texttt{spar()} and \texttt{spar.cv}) and \texttt{y} the vector of
  responses.
\item
  \texttt{update\_rpm\_w\_data} function for updating the random
  projection matrix with data. This can be used for the case where a
  list of random projection matrices is provided by argument
  \code{RPMs}. In this case, the random structure is kept fixed, but the
  data-dependent part gets updated with the provided data. Defaults to
  NULL. If not provided, the values of the provided RPMs do not change.
\item
  \texttt{control}, which is the control list in \texttt{screen\_*}.
  These controls are arguments which are needed in
  \texttt{generate\_scrcoef} in order to generate the desired screening
  coefficients.
\end{itemize}

Further arguments related to the random projection can be passed through
\texttt{...}, which will then be saved as attributes of the
``\texttt{randomprojection}'' object.

More specifically, the following are employed for all in
``\texttt{randomprojection}'' objects in function \texttt{spar()}:

\begin{itemize}
\item
  \texttt{mslow}: integer giving the minimum dimension to which the
  predictors should be projected; defaults to \(\log(p)\)
\item
  \texttt{msup}: integer giving the maximum dimension to which the
  predictors should be projected; defaults to \(n/2\)
\item
  \texttt{use\_data}: boolean indicating whether the random projection
  is data driven.
\end{itemize}

Note that for random projection matrices which satisfy the JL lemma,
\texttt{mslow} can be determined by employing the result of the JL
lemma, which gives a lower bound on the goal dimension in order to
preserve the distances between all pairs of points within a factor
\((1 \pm \epsilon)\): \(m>\frac{24}{3\epsilon^2-2\epsilon^3}\log n\).

For illustration purposes, consider the implemented function
\texttt{rp\_gaussian}, which is a random projection with entries draws
from the standard normal distribution.

\begin{verbatim}
rp_gaussian()
#> $name
#> [1] "rp_gaussian"
#> 
#> $generate_rp_fun
#> function(rp, m, included_vector) {
#>   p <- length(included_vector)
#>   vals <- rnorm(m * p)
#>   RM <- matrix(vals, nrow = m, ncol = p)
#>   RM <- Matrix(RM, sparse = TRUE)
#>   return(RM)
#> }
#> <environment: namespace:SPAR>
#> 
#> $update_data_fun
#> NULL
#> 
#> $control
#> list()
#> 
#> attr(,"use_data")
#> [1] FALSE
#> attr(,"class")
#> [1] "randomprojection"
\end{verbatim}

\begin{longtable}[]{@{}ll@{}}
\toprule\noalign{}
Name & Random projection method \\
\midrule\noalign{}
\endfirsthead
\toprule\noalign{}
Name & Random projection method \\
\midrule\noalign{}
\endhead
\bottomrule\noalign{}
\endlastfoot
\texttt{gaussian} & Standard Gaussian \\
\texttt{sparse} & Rademacher \\
\texttt{cw} & sparse embedding matrix \\
\texttt{cwdatadriven} & data driven sparse embedding matrix \\
\caption{Overview of implemented random projection
matrices.}\label{tbl-overviewrp}\tabularnewline
\end{longtable}

\subsection{Methods}\label{methods}

\begin{verbatim}
data("example_data")
spar_res <- spar(example_data$x, example_data$y,
                 xval = example_data$xtest,
                 yval = example_data$ytest,
                 nummods=c(5,10,15,20,25,30))
spar_res
#> SPAR object:
#> Smallest Validation Measure reached for nummod=30,
#>               nu=1.36e-02 leading to 1098 / 2000 active predictors.
#> Summary of those non-zero coefficients:
#>     Min.  1st Qu.   Median     Mean  3rd Qu.     Max. 
#> -0.84883 -0.06693  0.01612  0.02765  0.12026  1.06543
\end{verbatim}

Methods \texttt{print}, \texttt{plot}, \texttt{coef}, \texttt{predict}
are available for both ``\texttt{spar}'' and ``\texttt{spar.cv}''
classes.

The \texttt{print} method return information on

\section{Extensibility}\label{extensibility}

The user can implement their own screening and random projections.

\subsection{Screening coefficients}\label{screening-coefficients-1}

We exemplify how new screening coefficients implemented in package
\pkg{VariableScreening} can easily be used in the framework of
\pkg{SPAR}.

We start by defining the function for generating the screening
coefficients using the \texttt{screenIID()} function in
\pkg{VariableScreening}.

\begin{verbatim}
generate_scr_sirs <- function(scrcoef, data) {
  res_screen <- do.call(function(...) 
    VariableScreening::screenIID(data$x, data$y, ...), 
    scrcoef$control)
  coefs <- res_screen$measurement
  coefs
}
\end{verbatim}

Note that \texttt{screenIID()} also takes method as an argument. To
allow for flexibility, we do not fix the method in
\texttt{generate\_scr\_sirs} but rather allow the user to pass a method
through the \texttt{control} argument in the \texttt{screen\_*}
function. This function is created using
\texttt{constructor\_screeningcoef}:

\begin{verbatim}
screen_sirs <- constructor_screeningcoef(
  "screen_sirs", 
  generate_scrcoef = generate_scr_sirs)
\end{verbatim}

We now call the \texttt{spar()} function with the newly created
screening procedure. We consider method SIRS of \citet{zhu2011model},
which ranks the predictors by their correlation with the rank-ordered
response and we do not perform probabilistic variable screening but
employ the top \(2n\) variables in each marginal model.

\begin{verbatim}
set.seed(123)      
spar_example <- spar(example_data$x, example_data$y,
                     scrcoef = screen_sirs(type = "fixed",
                                           control=list(method = "SIRS")),
                     measure = "mse")
print(spar_example)
#> SPAR object:
#> Smallest Validation Measure reached for nummod=20,
#>               nu=1.75e-03 leading to 396 / 2000 active predictors.
#> Summary of those non-zero coefficients:
#>       Min.    1st Qu.     Median       Mean    3rd Qu.       Max. 
#> -0.6918230 -0.0928444  0.0009116  0.0943062  0.1777889  2.0089709
\end{verbatim}

\subsection{Random projections}\label{random-projections-1}

We exemplify how new random projections can be implemented in the
framework of \pkg{SPAR}.

We implement the random projection of \citet{cannings2017random}, who
propose using the Haar measure for generating the random projections.
They simulate matrices from Haar measure by independently drawing each
entry of a matrix \(Q\) from a standard normal distribution, and then to
take the projection matrix to be the transpose of the matrix of left
singular vectors in the singular value decomposition of \(Q\). Moreover,
they suggest using ``good'' random projections,

\begin{verbatim}
update_data_cannings <- function(rp, data) {
  attr(rp, "data") <- data
  rp
}
generate_cannings <- function(rp, m, included_vector) {
  p <- length(included_vector)
  if (is.null(rp$control$B2)) rp$control$B2 <- 50
  x <- attr(rp, "data")$x[, included_vector]
  y <- attr(rp, "data")$y
  
  ## Sample for all B2 cases 
  B2 <- rp$control$B2
  n <- nrow(x)
  id_test <- sample(n, size = n %/% 4)
  xtrain <- x[-id_test, ]
  xtest <- x[id_test,]
  ytrain <- y[-id_test]
  ytest <- y[id_test]
  ## Update with family argument of SPAR if not available
  if (is.null(rp$control$family)) {
    rp$control$family <- attr(rp, "family")
  }
  family <- rp$control$family
  control_glm <-
    rp$control[names(rp$control)  %in% names(formals(glm.fit))]

  error_all <- lapply(seq_len(B2), FUN = function(s){
    R0 <- matrix(1/sqrt(p) * rnorm(p * m), nrow = p, ncol = m)
    RM <- qr.Q(qr(R0))[, seq_len(m)]
    RM <- Matrix(t(RM), sparse = TRUE)
    xrp <- tcrossprod(xtrain, RM)
    mod <- do.call(function(...) 
      glm.fit(x =  cbind(1, xrp), y = ytrain, ...), control_glm)
    eta_test <- drop(cbind(1, tcrossprod(xtest, RM)) %*% mod$coefficients)
    pred <- family$linkinv(eta_test)
    if (family$family == "binomial") {
      pred <- (pred > 0.5) + 0
      out <- mean(pred != ytest)
    } else {
      out <- mean((pred - ytest)^2)
    }
    list(RM, out)
  })
  id_best <- which.min(sapply(error_all, "[[", 2))
  RM <- error_all[[id_best]][[1]]
  return(RM)
}

rp_cannings <- constructor_randomprojection(
  "rp_cannings",
  generate_fun = generate_cannings,
  update_data_fun = update_data_cannings
)
\end{verbatim}

We can now estimate SPAR

\begin{verbatim}
set.seed(123)      
ystar <- ( example_data$y > 0 ) + 0
ystarval <- ( example_data$ytest > 0 ) + 0
spar_example_1 <- spar(
  x=example_data$x, y=ystar,
  xval = example_data$xtest, yval = ystarval,
  family = binomial(),
  nummods = 100, 
  scrcoef = screen_marglik(type = "fixed"),
  rp = rp_cannings(control = list(B2 = 50))
)

spar_example_2 <- spar(x = example_data$x, y = ystar,
  family = binomial(),
  scrcoef = screen_marglik(type = "fixed"),
  rp = rp_cw(data = TRUE),
  nummods = 100, 
  xval = example_data$xtest, yval = ystarval
)
spar_example_1$val_res
spar_example_2$val_res
\end{verbatim}

\section{Illustrations}\label{sec-illustrations}

\subsection{Face image data}\label{face-image-data}

We illustrate the package on a data set containing \(n = 698\) black and
white face images of size \(p = 64 \times 64 = 4096\) and the faces'
horizontal looking direction angle as the response variable. The Isomap
face data can be found online on
https://web.archive.org/web/20160913051505/http://isomap.
stanford.edu/datasets.html

\begin{verbatim}
library("R.matlab")
temp <- tempdir()
download.file("https://web.archive.org/web/20150922051706/http://isomap.stanford.edu/face_data.mat.Z", file.path(temp, "face_data.mat.Z"))
system(sprintf('uncompress %s', paste0(temp, "/face_data.mat.Z")))
facedata <- readMat(file.path(temp, "face_data.mat"))

x <- t(facedata$images)
y <- facedata$poses[1,]
\end{verbatim}

We can visualize e.g., the first observation in this data set by:

\begin{verbatim}
library(ggplot2)
i <- 1
ggplot(data.frame(X = rep(1:64,each=64),Y = rep(64:1,64),
                  Z = facedata$images[,i]),
       aes(X, Y, fill = Z)) +
  geom_tile() +
  theme_void() +
  ggtitle(paste0("y = ",round(facedata$poses[1,i],1))) +
  theme(legend.position = "none",
        plot.title = element_text(hjust = 0.5))
\end{verbatim}

We can split the data into training vs test sample:

\begin{verbatim}
set.seed(1234)
ntot <- length(y)
ntest <- ntot * 0.25
testind <- sample(1:ntot, ntest, replace=FALSE)
xtrain <- as.matrix(x[-testind, ])
ytrain <- y[-testind]
xtest <- as.matrix(x[testind, ])
ytest <- y[testind]
\end{verbatim}

We can now estimate the model on the training data:

\begin{verbatim}
library(SPAR)
spar_faces <- spar.cv(xtrain, ytrain,
                      family = gaussian(),
                      nummods = c(5, 10, 20, 50),
                      type.measure = "mse")
spar_faces
\end{verbatim}

The \texttt{plot} method for `\texttt{spar.cv}' objects displays by
default the measure employed in the cross validation (in this case MSE)
for a grid of \(\lambda\) values, where the number of models is fixed to
the value found to perform best in cross-validation exercise:

\begin{verbatim}
plot(spar_faces)
\end{verbatim}

The coefficients of the different variables (in this example pixels)
obtained by averaging over the coefficients the marginal models (for
optimal \(\lambda\) and number of models) are given by:

\begin{verbatim}
face_coef <- coef(spar_faces, opt_par = "best")
str(face_coef)
\end{verbatim}

The coefficients from each of the marginal models (before averaging) can
be plotted using the \texttt{plot(...,\ plot\_type\ =\ "coefs")}

\begin{verbatim}
plot(spar_faces, "coefs")
\end{verbatim}

The \texttt{predict()} function can be applied to the `\texttt{spar.cv}'
object:

\begin{verbatim}
ynew <- predict(spar_faces, xnew = xtest)
\end{verbatim}

In the high-dimensional setting it is interesting to look at the
relative mean square prediction error which compares the MSE to the MSE
of a model containing only an intercept:

\begin{verbatim}
rMSPEconst <- mean((ytest - mean(y))^2) 
mean((ynew-ytest)^2)/rMSPEconst
\end{verbatim}

Additionally, for this data set, one can visualize the effect of each
pixel \(\hat\beta_j x^\text{new}_{i,j}\) in predicting the face
orientation in a given image e.g., 9th in the test set:

\begin{verbatim}
i <- 9
plot4 <- ggplot(data.frame(X = rep(1:64, each = 64),
                           Y = rep(64:1, 64),
                           effect = xtest[i,] * face_coef$beta), 
                aes(X, Y, fill= effect)) +
  geom_tile() +
  theme_void() +
  scale_fill_gradient2() +
  ggtitle(bquote(hat(y) == .(round(ynew[i])))) +
  theme(plot.title = element_text(hjust = 0.5)) 
plot4
\end{verbatim}

\subsection{Darwin data set}\label{darwin-data-set}

The Darwin dataset \citep{CILIA2022darwin} contains a binary response
for Alzheimer's disease (AD) together with extracted features from 25
handwriting tests (18 features per task) for 89 AD patients and 85
healthy people (\(n=174\)).

The data set can be downloaded from
https://archive.ics.uci.edu/dataset/732/darwin:

\begin{verbatim}
temp <- tempfile()
download.file("https://archive.ics.uci.edu/static/public/732/darwin.zip", temp)
darwin_tmp <- read.csv(unzip(temp,  "data.csv"), stringsAsFactors = TRUE)
\end{verbatim}

Before proceeding with the analysis, the data is screened for
multivariate outliers using the DDC algorithm in package \pkg{cellWise}.

\begin{verbatim}
darwin_orig <- list(
  x = as.matrix(darwin_tmp[, !(colnames(darwin_tmp) %in% c("ID", "class"))]),
  y = as.numeric(darwin_tmp$class) - 1)

tmp <- cellWise::DDC(darwin_orig$x,
                     list(returnBigXimp = TRUE, 
                          tolProb = 0.999,
                          silent = TRUE))
darwin <- list(x = tmp$Ximp,
               y = darwin_orig$y)
\end{verbatim}

We estimate the spar model with binomial family and logit link and use
\(1-\)area under the ROC curve as the cross-validation measure:

\begin{verbatim}
spar_darwin <- spar.cv(darwin$x, darwin$y,
                       family = binomial(logit),
                       nummods = c(5, 10, 20, 50),
                       type.measure = "1-auc")
\end{verbatim}

The \texttt{plot} method for `\texttt{spar.cv}' objects displays by
default the measure employed in the cross validation (in this case MSE)
for a grid of \(\lambda\) values, where the number of models is fixed to
the value found to perform best in cross-validation exercise:

\begin{verbatim}
plot(spar_darwin)
\end{verbatim}

The plot of the coefficients can be interpreted nicely in this example:

\begin{verbatim}
ntasks <- 25
nfeat <- 18
reorder_ind <- c(outer((seq_len(ntasks) - 1) * nfeat, seq_len(nfeat), "+"))
feat_names <- sapply(colnames(darwin$x)[seq_len(nfeat)],
                     function(name) substr(name, 1, nchar(name) - 1))

plot(spar_darwin,"coefs",coef_order = reorder_ind) + 
  geom_vline(xintercept = 0.5 + seq_len(ntasks - 1) * ntasks, 
             alpha = 0.2, linetype = 2) +
  annotate("text",x = (seq_len(nfeat) - 1) * ntasks + 12,
           y = 45,label=feat_names, angle = 90,
           size = 3)
\end{verbatim}

In general we observe that the different features measures across
different tasks have the same impact on the probability of AD
(observable by the blocks of blue or red lines).

\section{Conclusion}\label{sec-conclusion}

Package \pkg{SPAR} provides an implementation for estimating an ensemble
of GLMs after performing probabilistic screening and random projection
in a high-dimensional setting.

\section*{Computational details}\label{computational-details}

The results in this paper were obtained using \proglang{R} 4.4.0.

\proglang{R} itself and all packages used are available from the
Comprehensive \proglang{R} Archive Network (CRAN) at
\url{https://CRAN.R-project.org/}.

\section*{Acknowledgments}\label{acknowledgments}

Roman Parzer and Laura Vana-Gür acknowledge funding from the Austrian
Science Fund (FWF) for the project ``High-dimensional statistical
learning: New methods to advance economic and sustainability policies''
(ZK 35), jointly carried out by WU Vienna University of Economics and
Business, Paris Lodron University Salzburg, TU Wien, and the Austrian
Institute of Economic Research (WIFO).


\renewcommand\refname{References}
  \bibliography{SPAR.bib}



\end{document}
